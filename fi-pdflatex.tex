%%%%%%%%%%%%%%%%%%%%%%%%%%%%%%%%%%%%%%%%%%%%%%%%%%%%%%%%%%%%%%%%%%%%
%% I, the copyright holder of this work, release this work into the
%% public domain. This applies worldwide. In some countries this may
%% not be legally possible; if so: I grant anyone the right to use
%% this work for any purpose, without any conditions, unless such
%% conditions are required by law.
%%%%%%%%%%%%%%%%%%%%%%%%%%%%%%%%%%%%%%%%%%%%%%%%%%%%%%%%%%%%%%%%%%%%

\documentclass[
  digital, %% This option enables the default options for the
           %% digital version of a document. Replace with `printed`
           %% to enable the default options for the printed version
           %% of a document.
  oneside, %% This option enables double-sided typesetting. Use at
           %% least 120 g/m² paper to prevent show-through. Replace
           %% with `oneside` to use one-sided typesetting; use only
           %% if you don’t have access to a double-sided printer,
           %% or if one-sided typesetting is a formal requirement
           %% at your faculty.
  table,   %% This option causes the coloring of tables. Replace
           %% with `notable` to restore plain LaTeX tables.
  lof,     %% This option prints the List of Figures. Replace with
           %% `nolof` to hide the List of Figures.
  lot,     %% This option prints the List of Tables. Replace with
           %% `nolot` to hide the List of Tables.
  %% More options are listed in the user guide at
  %% <http://mirrors.ctan.org/macros/latex/contrib/fithesis/guide/mu/fi.pdf>.
]{fithesis3}
%% The following section sets up the locales used in the thesis.
\usepackage[resetfonts]{cmap} %% We need to load the T2A font encoding
\usepackage[T1,T2A]{fontenc}  %% to use the Cyrillic fonts with Russian texts.
\usepackage[
  main=english, %% By using `czech` or `slovak` as the main locale
                %% instead of `english`, you can typeset the thesis
                %% in either Czech or Slovak, respectively.
  english, german, russian, czech, slovak %% The additional keys allow
]{babel}        %% foreign texts to be typeset as follows:
%%
%%   \begin{otherlanguage}{german}  ... \end{otherlanguage}
%%   \begin{otherlanguage}{russian} ... \end{otherlanguage}
%%   \begin{otherlanguage}{czech}   ... \end{otherlanguage}
%%   \begin{otherlanguage}{slovak}  ... \end{otherlanguage}
%%
%% For non-Latin scripts, it may be necessary to load additional
%% fonts:
\usepackage{paratype}
\def\textrussian#1{{\usefont{T2A}{PTSerif-TLF}{m}{rm}#1}}
%%
%% The following section sets up the metadata of the thesis.
\thesissetup{
    date          = \the\year/\the\month/\the\day,
    university    = mu,
    faculty       = fi,
    type          = bc,
    author        = Marek Mravík,
    gender        = m,
    advisor       = Mgr. Lukáš Němec,
    title         = {Virtual assistant},
    TeXtitle      = {Virtual assistant},
    keywords      = {virtual, assistant, AI, speech recognition, text                 to speech, machine learning, deep learning, Amazon, Alexa, Google, Apple, Siri},
    TeXkeywords   = {virtual, assistant, AI, speech recognition, text                 to speech, machine learning, deep learning, Amazon, Alexa, Google, Apple, Siri},
    abstract      = {This thesis is focused on virtual assistants, their history, availability and comparison. For purpose of this thesis I have selected Amazon Alexa, Google Assistant and Apple Siri. Security, speech-to-text understanding and functionality of these assistants will be analyzed and tested. Second part of the thesis defines how a virtual assistant should work, behave and be designed at my opinion and a really simple basic assistant with few functions is implemented.},
    thanks        = {These are the acknowledgements for my thesis, which can

                     span multiple paragraphs.},
    bib           = example.bib,
}
\usepackage{makeidx}      %% The `makeidx` package contains
\makeindex                %% helper commands for index typesetting.
%% These additional packages are used within the document:
\usepackage{paralist} %% Compact list environments
\usepackage{amsmath}  %% Mathematics
\usepackage{amsthm}
\usepackage{amsfonts}
\usepackage{url}      %% Hyperlinks
\usepackage{markdown} %% Lightweight markup
\usepackage{listings} %% Source code highlighting
\lstset{
  basicstyle      = \ttfamily,%
  identifierstyle = \color{black},%
  keywordstyle    = \color{blue},%
  keywordstyle    = {[2]\color{cyan}},%
  keywordstyle    = {[3]\color{olive}},%
  stringstyle     = \color{teal},%
  commentstyle    = \itshape\color{magenta}}
\usepackage{floatrow} %% Putting captions above tables
\floatsetup[table]{capposition=top}




\begin{document}
\chapter*{Introduction}
\addcontentsline{toc}{chapter}{Introduction}
In the current era, people are stressed out from their fast-living stereotypical lives. People these days do not have enough time for themselves, having too many responsibilities at work, spending a lot of time in transports, shopping... The goal of this era is to make everything more efficient, simpler, faster, etc. One might think that people just got lazy over time, but why would we do something if we can make something do it for us. Thanks to this, and IT progress, we have so many virtual assistants implementations available. Someone not interested in this topic might have never heard about this so-called virtual assistant. So let's define what it is first. A virtual assistant can also be called digital or AI assistant, is a piece of software, an application program, that can talk to the user, understand what he says in natural language. Yes, the principle is basically talking to a computer or whatever device an assistant is running on. The user can command its assistant, and the assistant completes tasks for its user. Simple right? So what can a virtual assistant do for a user? Nowadays pretty much everything. Starting in the morning, wishing the user to have a beautiful day, giving him the newest info right away, informing him about the weather, whatever user prefers. It can also keep an eye on the users shopping list, to-do list, notes, giving user everything he wants to know about users stuff anytime he asks, adding items just by saying it. The assistant can keep the collection of users favorite music, artists, play what user says on demand (this feature might need other hardware). Virtual assistants are taking workload from secretaries, or personal assistants as well since they can read users new emails to the user, notify the user about anything that happened while the user was not available, create and send new emails just by dictating them.

How do virtual assistants work and how are they able to do such things? How do virtual assistants even work? Simply said, virtual assistants are recording anything user says, transforming it to something they can work with, analyzing users request, preparing the answer or action, applying the operation or giving the user what he asked for, and if needed, telling the user that the work has been done. Most of the virtual assistants are cloud-based, which means that they perform just about everything on remote high power servers. The essential functions that are necessary for the process are speech-to-text and text-to-speech transformations, and a lot of AI platforms based on machine learning, deep learning, neural networks which are somewhat like a brain for these assistants. This brain or brains are developed every day, with every interaction with a user, with lots of sophisticated, specialized algorithms to learn from their mistakes and to make better predictions and understanding of users demands.

\chapter{History}
The concept of virtual assistants is not as young as many people think. It has started developing in early 1960. The first to introduce elementary voice assistant was the Big Blue company, also known as IBM. IBM's \texttt{Shoebox} device set the start of this long-time developing competition between giant IT companies. At that time the Shoebox device was able to understand 16 words and ten digits. The shoebox was operated by speaking into a microphone, which converted voice sounds into electrical impulses and instructed an adding machine to calculate and print answers to simple arithmetic problems \parencite{shoebox}.

Another breaking point in the history of virtual assistants, or at that time just speech recognition software, was when Dragon launched the first speech recognition software, that was available for consumers for only a few thousands of dollars.

Next milestone was set by Microsoft when it introduced \texttt{Clippy}. Clippy was an intelligent user assistant implemented as a feature to Microsoft Office. Clippy had a form of a cartoon clip character, which offered users way too much help than was needed. When Clippy first came out, there was no option to turn it off, and it was more of a problem than a help. Even after the feature was able to turn off, it didn't live up to its purpose and disappeared in 2001 \parencite{clippy}.

Near the end of 00's decade, and the start of the next one, new virtual assistant in the shape as we know them today were introduced. Apple introduced its Speech Interpretation and Recognition Interface known as \texttt{Siri}. Followed by Google introducing \texttt{Google Now}, Microsoft with its \texttt{Cortana} and Amazon introducing \texttt{Alexa} and \texttt{Amazon Echo} just for users with Prime subscription.

These virtual assistants were not as we knew them when they have got introduced, but they got to their actual shape by never-ending evolution. This evolution started what looks to be the biggest competition of today's largest companies to have the best, most accurate and powerful AI assistant in the world.

\chapter{Available virtual assistants}
As I already mentioned, there are few assistants that most people heard of previously. Those are Apple Siri, Google Assistant, Amazon Alexa, Microsoft Cortana. We will talk about the first three of these later.

The market of virtual assistants is enormous, and we can see small companies or even groups of people trying to compete with the giants as Amazon or Apple. Many of AI assistant implementations are expensive, and we have to pay for them. There are some open-source, free implementations for anyone. 

One of the open-source voice assistants is \texttt{Mycroft}. Mycroft can run pretty much anywhere the user wants, on a desktop computer, in your car, or even on Raspberry Pi. As it is open-source software, the user, if he knows how can extend its features and improve it\parencite{mycroft}.

Another example of a free virtual assistant is \texttt{Braina}. Braina is an intelligent personal assistant for Windows PC. Braina also supports multiple languages and has a unique application for Android or iOS. With this mobile application you can interact with your computer you have Braina installed on anywhere in your house\parencite{braina}.

There is a lot of other companies working on their virtual assistants like Samsung Electronics, Blackberry Limited or even Facebook has their virtual assistant called M.

Virtual assistants are there, around all of us, waiting to get noticed, tried out and becoming a part of our lives. If this thesis gets you on board and you will be interested in getting one of your virtual assistants, go and look for one suited for you.


\chapter{Amazon Alexa}
\section{About}

Amazon Alexa was first introduced in November 2014 alongside Amazon Echo. At that time it was primarily a smart speaker with an option to control your music with voice. Alexa was the virtual agent that was powering the speaker and since then evolved into a center of a smart home. Alexa was inspired by talking computer systems that begin to appear in Star Trek. Alexa got its name after the ancient library of Alexandria\parencite{alexa_wiki}.

Alexa's transformation into a center of a smart home from pure music playing gadget wasn't as fast as you could think and even its developers did not think about this future use of Alexa. As the smart home devices came to market before Alexa and Echo, at times that these home assistants did not exist, there was no simple way to use them, unless you had a controller or your smartphone. The manufacturers of these smart home devices started looking for something that would unite and simplify the use of their products and thought that Alexa could be the resolution to their problems since it was there in people's living room right next to their devices, so they started calling the developers of Alexa with offers to cooperation. Like this Alexa slowly got to its shape as we know it today\parencite{alexa_businessinsider}.

It is running with a natural-language processing system, which is controlled just by talking to it. Alexa, like many other assistants, is always listening to what you say, waiting for her time to wake up, and do the commands you say or answer your questions. By default, to wake Alexa up, you have to call it by name "Alexa" and then you can follow with your question or command. This "Alexa" word is called 'wake-up word.' User can also change it to "Computer," "Echo" or "Amazon." 

The listening system, if you own Amazon Echo, is powered by many independent, sensitive microphones, which give Alexa the ability to hear you across the room loud and clear. When you wake Alexa up with the wake-up word, it records anything you say and sends it to Amazon's cloud computers which analyze the recording. Depending on your request the appropriate response is made. From playing music, you requested (if you have a subscription for music streaming platform compatible with Alexa) through answering you any current weather conditions you ask for to operating your smart products as thermostats, lights, computers or consoles, pretty much anything connected to Alexa\parencite{alexa_wirecutter}.

\section{Security}
\section{Tests}
\section{Evaluation}

\chapter{Google Assistant}
\section{About}

Google Assistant was first introduced as part of Google Home smart speaker at Google's developer conference that took place in May 2016. At first, it was just a part of a chat application, and it was far from perfect. As a proof of special treatment to development of Google Assistant is that they hired ex-Pixar animator Emma Coats to create and give the Assistant a little more personality.

Before the first appearance on Android smartphones running Android 6.0 Marshmallow or Android 7.0 Nougat in February 2017, it was exclusive to Pixel and Pixel NL smartphones. After that Google Assistant slowly found its way to Android Wear 2.0, Android TVs, laptops, cars and tablets. Google Assistant was also released to iOS operating system devices as a stand-alone app in 2017.

\section{Security}
\section{Tests}
\section{Evaluation}

\chapter{Apple Siri}
\section{About}
\section{Security}
\section{Tests}
\section{Evaluation}

\chapter{Comparison}

\chapter{These are}
\section{the available}
\subsection{sectioning}
\subsubsection{commands.}
\paragraph{Paragraphs and}
\subparagraph{subparagraphs are available as well.}
Inside the text, you can also use unnumbered lists,
\begin{itemize}
  \item such as
  \item this one
  \begin{itemize}
    \item     and they can be nested as well.
    \item[>>] You can even turn the bullets into something fancier,
    \item[\S] if you so desire.
  \end{itemize}
\end{itemize}
Numbered lists are
\begin{enumerate}
  \item very
  \begin{enumerate}
    \item similar
  \end{enumerate}
\end{enumerate}
and so are description lists:
\begin{description}
  \item[Description list]
    A list of terms with a description of each term
\end{description}
The spacing of these lists is geared towards paragraphs of text.
For lists of words and phrases, the \textsf{paralist} package
offers commands
\begin{compactitem}
  \item that
  \begin{compactitem}
    \item are
    \begin{compactitem}
      \item better
      \begin{compactitem}
        \item suited
      \end{compactitem}
    \end{compactitem}
  \end{compactitem}
\end{compactitem}
\begin{compactenum}
  \item to
  \begin{compactenum}
    \item this
    \begin{compactenum}
      \item kind of
      \begin{compactenum}
        \item content.
      \end{compactenum}
    \end{compactenum}
  \end{compactenum}
\end{compactenum}
The \textsf{amsthm} package provides the commands necessary for the
typesetting of mathematical definitions, theorems, lemmas and
proofs.

%% We will define several mathematical sectioning commands.
\newtheorem{theorem}{Theorem}[section] %% The numbering of theorems
                               %% will be reset after each section.
\newtheorem{lemma}[theorem]{Lemma}         %% The numbering of lemmas
\newtheorem{corollary}[theorem]{Corollary} %% and corollaries will
                               %% share the counter with theorems.
\theoremstyle{definition}
\newtheorem{definition}{Definition}
\theoremstyle{remark}
\newtheorem*{remark}{Remark}

\begin{theorem}
  This is a theorem that offers a profound insight into the
  mathematical sectioning commands.
\end{theorem}
\begin{theorem}[Another theorem]
  This is another theorem. Unlike the first one, this theorem has
  been endowed with a name.
\end{theorem}
\begin{lemma}
  Let us suppose that $x^2+y^2=z^2$. Then
  \begin{equation}
    \biggl\langle u\biggm|\sum_{i=1}^nF(e_i,v)e_i\biggr\rangle
    =F\biggl(\sum_{i=1}^n\langle e_i|u\rangle e_i,v\biggr).
  \end{equation}
\end{lemma}
\begin{proof}
  $\nabla^2 f(x,y)=\frac{\partial^2f}{\partial x^2}+
   \frac{\partial^2f}{\partial y^2}$.
\end{proof}
\begin{corollary}
  This is a corollary.
\end{corollary}
\begin{remark}
  This is a remark.
\end{remark}

\chapter{Floats and references}
\begin{figure}
  \begin{center}
    %% PNG and JPG images can be inserted into the document as well,
    %% but their resolution needs to be adequate. The minimum is
    %% about 100 pixels per 1 centimeter or 300 pixels per 1 inch.
    %% That means that a JPG or PNG image typeset at 4 × 4 cm should
    %% be 400 × 400 px large at the bare minimum.
    %%
    %% The optimum is about 250 pixels per 1 centimeter or 600
    %% pixels per 1 inch. That means that a JPG or PNG image typeset
    %% at 4 × 4 cm should be 1000 × 1000 px large or larger.
    \includegraphics[width=4cm]{fithesis/logo/mu/fithesis-base.pdf}
  \end{center}
  \caption{The logo of the Masaryk University at 40\,mm}
  \label{fig:mulogo1}
\end{figure}

\begin{figure}
  \begin{center}
    \begin{minipage}{.66\textwidth}
      \includegraphics[width=\textwidth]{fithesis/logo/mu/fithesis-base.pdf}
    \end{minipage}
    \begin{minipage}{.33\textwidth}
      \includegraphics[width=\textwidth]{fithesis/logo/mu/fithesis-base.pdf} \\
      \includegraphics[width=\textwidth]{fithesis/logo/mu/fithesis-base.pdf}
    \end{minipage}
  \end{center}
  \caption{The logo of the Masaryk University at $\frac23$ and
    $\frac13$ of text width}
  \label{fig:mulogo2}
\end{figure}

\begin{table}
  \begin{tabularx}{\textwidth}{lllX}
    \toprule
    Day & Min Temp & Max Temp & Summary \\
    \midrule
    Monday & $13^{\circ}\mathrm{C}$ & $21^\circ\mathrm{C}$ & A
    clear day with low wind and no adverse current advisories. \\
    Tuesday & $11^{\circ}\mathrm{C}$ & $17^\circ\mathrm{C}$ & A
    trough of low pressure will come from the northwest. \\
    Wednesday & $10^{\circ}\mathrm{C}$ &
    $21^\circ\mathrm{C}$ & Rain will spread to all parts during the
    morning. \\
    \bottomrule
  \end{tabularx}
  \caption{A weather forecast}
  \label{tab:weather}
\end{table}

The logo of the Masaryk University is shown in Figure
\ref{fig:mulogo1} and Figure \ref{fig:mulogo2} at pages
\pageref{fig:mulogo1} and \pageref{fig:mulogo2}. The weather
forecast is shown in Table \ref{tab:weather} at page
\pageref{tab:weather}. The following chapter is Chapter
\ref{chap:matheq} and starts at page \pageref{chap:matheq}.
Items \ref{item:star1}, \ref{item:star2}, and
\ref{item:star3} are starred in the following list:
\begin{compactenum}
  \item some text
  \item some other text
  \item $\star$ \label{item:star1}
  \begin{compactenum}
    \item some text
    \item $\star$ \label{item:star2}
    \item some other text
    \begin{compactenum}
      \item some text
      \item some other text
      \item yet another piece of text
      \item $\star$ \label{item:star3}
    \end{compactenum}
    \item yet another piece of text
  \end{compactenum}
  \item yet another piece of text
\end{compactenum}
If your reference points to a place that has not yet been typeset,
the \verb"\ref" command will expand to \textbf{??} during the first
run of
\texttt{pdflatex \jobname.tex}
and a second run is going to be needed for the references to
resolve. With online services -- such as Overleaf -- this is
performed automatically.

\chapter{Mathematical equations}
\label{chap:matheq}
\TeX{} comes pre-packed with the ability to typeset inline
equations, such as $\mathrm{e}^{ix}=\cos x+i\sin x$, and display
equations, such as \[
  \mathbf{A}^{-1} = \begin{bmatrix}
  a & b \\ c & d \\
  \end{bmatrix}^{-1} =
  \frac{1}{\det(\mathbf{A})} \begin{bmatrix}
  \,\,\,d & \!\!-b \\ -c & \,a \\
  \end{bmatrix} =
  \frac{1}{ad - bc} \begin{bmatrix}
  \,\,\,d & \!\!-b \\ -c & \,a \\
  \end{bmatrix}.
\] \LaTeX{} defines the automatically numbered \texttt{equation}
environment:
\begin{equation}
  \gamma Px = PAx = PAP^{-1}Px.
\end{equation}
The package \textsf{amsmath} provides several additional
environments that can be used to typeset complex equations:
\begin{enumerate}
  \item An equation can be spread over multiple lines using the
    \texttt{multline} environment:
    \begin{multline}
      a + b + c + d + e + f + b + c + d + e + f + b + c + d + e +
f \\
      + f + g + h + i + j + k + l + m + n + o + p + q
    \end{multline}

  \item Several aligned equations can be typeset using the
    \texttt{align} environment:
    \begin{align}
              a + b &= c + d     \\
                  u &= v + w + x \\[1ex]
      i + j + k + l &= m
    \end{align}

  \item The \texttt{alignat} environment is similar to
    \texttt{align}, but it doesn't insert horizontal spaces between
    the individual columns:
    \begin{alignat}{2}
      a + b + c &+ d       &   &= 0 \\
              e &+ f + g   &   &= 5
    \end{alignat}

  \item Much like chapter, sections, tables, figures, or list
    items, equations -- such as \eqref{eq:first} and
    \eqref{eq:mine} -- can also be labeled and referenced:
    \begin{alignat}{4}
      b_{11}x_1 &+ b_{12}x_2  &  &+ b_{13}x_3  &  &             &
        &= y_1,                   \label{eq:first} \\
      b_{21}x_1 &+ b_{22}x_2  &  &             &  &+ b_{24}x_4  &
        &= y_2. \tag{My equation} \label{eq:mine}
    \end{alignat}

  \item The \texttt{gather} environment makes it possible to
    typeset several equations without any alignment:
    \begin{gather}
      \psi = \psi\psi, \\
      \eta = \eta\eta\eta\eta\eta\eta, \\
      \theta = \theta.
    \end{gather}

  \item Several cases can be typeset using the \texttt{cases}
    environment:
    \begin{equation}
      |y| = \begin{cases}
        \phantom-y & \text{if }z\geq0, \\
                -y & \text{otherwise}.
      \end{cases}
    \end{equation}
\end{enumerate}
For the complete list of environments and commands, consult the
\textsf{amsmath} package manual\footnote{
  See \url{http://mirrors.ctan.org/macros/latex/required/amslatex/math/amsldoc.pdf}.
  The \texttt{\textbackslash url} command is provided by the
  package \textsf{url}.
}.

\chapter{\textnormal{We \textsf{have} \texttt{several} \textsc{fonts}
  \textit{at} \textbf{disposal}}}
The serified roman font is used for the main body of the text.
\textit{Italics are typically used to denote emphasis or
quotations.} \texttt{The teletype font is typically used for source
code listings.} The \textbf{bold}, \textsc{small-caps} and
\textsf{sans-serif} variants of the base roman font can be used to
denote specific types of information.

\tiny We \scriptsize can \footnotesize also \small change \normalsize
the \large font \Large size, \LARGE although \huge it \Huge
is \huge usually \LARGE not \Large necessary.\normalsize

A wide variety of mathematical fonts is also available, such as: \[
  \mathrm{ABC}, \mathcal{ABC}, \mathbf{ABC}, \mathsf{ABC},
  \mathit{ABC}, \mathtt{ABC}
\] By loading the \textsf{amsfonts} packages, several additional
fonts will become available: \[
  \mathfrak{ABC}, \mathbb{ABC}
\] Many other mathematical fonts are available\footnote{
  See \url{http://tex.stackexchange.com/a/58124/70941}.
}.

\chapter{Using lightweight markup}
\shorthandoff{-}
\begin{markdown*}{%
  hybrid,
  definitionLists,
  footnotes,
  inlineFootnotes,
  hashEnumerators,
  fencedCode,
  citations,
  citationNbsps,
}

If you decide that \LaTeX{} is too wordy for some parts of your
document, there are [packages](https://www.ctan.org/pkg/markdown
"Markdown") that allow you to use more lightweight markup next
to it.

 ![logo](fithesis/logo/mu/fithesis-base.pdf "The logo of the
  Masaryk University")

This is a bullet list. Unlike numbered lists, bulleted lists
contain an **unordered** set of bullet points. When a bullet point
contains multiple paragraphs, the list is typeset as follows:

  * The first item of a bullet list

    that spans several paragraphs,
  * the second item of a bullet list,
  * the third item of a bullet list.

When none of the bullet points contains multiple paragraphs, the
list has a more compact form:

  * The first item of a bullet list,
  * the second item of a bullet list,
  * the third item of a bullet list.

Unlike a bulleted list, a numbered list implies chronology or
ordering of the bullet points. When a bullet point
contains multiple paragraphs, the list is typeset as follows:

  1. The first item of an ordered list

     that spans several paragraphs,
  2. the second item of an ordered list,
  3. the third item of an ordered list.
  #. If you are feeling lazy,
  #. you can use hash enumerators as well.

When none of the bullet points contains multiple paragraphs, the
list has a more compact form:

  6. The first item of an ordered list,
  7. the second item of an ordered list,
  8. the third item of an ordered list.

Definition lists are used to provide definitions of terms. When
a definition contains multiple paragraphs, the list is typeset
as follows:

Term 1

:   Definition 1

*Term 2*

:   Definition 2

        Some code, part of Definition 2

    Third paragraph of Definition 2.

When none of the bullet points contains multiple paragraphs, the
list has a more compact form:

Term 1
:   Definition 1
*Term 2*
:   Definition 2

Block quotations are used to include an excerpt from an external
document in way that visually clearly separates the excerpt from
the rest of the work:

> This is the first level of quoting.
>
> > This is nested blockquote.
>
> Back to the first level.

Footnotes are used to include additional information to the
document that are not necessary for the understanding of the main
text. Here is a footnote reference^[Here is the footnote.] and
another.[^longnote]

[^longnote]: Here's one with multiple blocks.

    Subsequent paragraphs are indented to show that they
belong to the previous footnote.

        Some code

    The whole paragraph can be indented, or just the first
    line.  In this way, multi-paragraph footnotes work like
    multi-paragraph list items.

Citations are used to provide bibliographical references to other
documents. This is a regular citation~[@borgman03, p. 123]. This is
an in-text citation: @borgman03\. You can also cite several authors
at once using both regular~[see @borgman03, p. 123; @greenberg98,
sec.  3.2; and @thanh01] and in-text citations: @borgman03 [p.123;
@greenberg98, sec. 3.2; @thanh01].

Code blocks are used to include source code listings into the
document:

    #include <stdio.h>
    #include <unistd.h>
    #include <sys/types.h>
    #include <sys/wait.h>
    // This is a comment
    int main(int argc, char **argv)
    {
        while (--c > 1 && !fork());
        sleep(c = atoi(v[c]));
        printf("%d\n", c);
        wait(0);
        return 0;
    }

There is an alternative syntax for code blocks that allows you to
specify additional information, such as the language of the source
code. This information can be used for syntax highlighting:

``` sh
#!/bin/sh
fac() {
  if [ "$1" -leq 1 ]; then
    echo 1
  else
    echo $(("$1" * fac $(("$1" - 1))))
  fi
}
``````````````

~~~~~~ Ruby
# Here's a way to empty an array.
joe = [ 'eggs.', 'some', 'break', 'to', 'Have' ]
print(joe.pop, " ") while joe.size > 0
print "\n"
~~~~~~

\end{markdown*}
\shorthandon{-}

\chapter{Inserting the bibliography}
After linking a bibliography data\-base files to the document using
the \verb"\"\texttt{thesis\discretionary{-}{}{}setup\{bib\discretionary{=}{=}{=}%
\{\textit{file1},\textit{file2},\,\ldots\,\}\}} command, you can
start citing the entries. This is just dummy text
\parencite{borgman03} lightly sprinkled with citations
\parencite[p.~123]{greenberg98}. Several sources can be cited at
once: \cite{borgman03,greenberg98,thanh01}.
\citetitle{greenberg98} was written by \citeauthor{greenberg98} in
\citeyear{greenberg98}. We can also produce \textcite{greenberg98}%
\ or %% Let us define a compound command:
\def\citeauthoryear#1{(\textcite{#1},~\citeyear{#1})}%
\citeauthoryear{greenberg98}%
. The full bibliographic citation is:
\emph{\fullcite{greenberg98}}. We can easily insert a bibliographic
citation into the footnote\footfullcite{greenberg98}.

The \verb"\nocite" command will not generate any
output\nocite{muni}, but it will insert its arguments into
the bibliography. The \verb"\nocite{*}" command will insert all the
records in the bibliography database file into the bibliography.
Try uncommenting the command
%% \nocite{*}
and watch the bibliography section come apart at the seams.

When typesetting the document for the first time, citing a
\texttt{work} will expand to [\textbf{work}] and the
\verb"\printbibliography" command will produce no output. It is now
necessary to generate the bibliography by running \texttt{biber
\jobname.bcf} from the command line and then by typesetting the
document again twice. During the first run, the bibliography
section and the citations will be typeset, and in the second run,
the bibliography section will appear in the table of contents.

The \texttt{biber} command needs to be executed from within the
directory, where the \LaTeX\ source file is located. In Windows,
the command line can be opened in a directory by holding down the
\textsf{Shift} key and by clicking the right mouse button while
hovering the cursor over a directory.  Select the \textsf{Open
Command Window Here} option in the context menu that opens shortly
afterwards.

With online services -- such as Overleaf -- or when using an
automatic tool -- such as \LaTeX MK -- all commands are executed
automatically. When you omit the \verb"\printbibliography" command,
its location will be decided by the template.

  \printbibliography[heading=bibintoc] %% Print the bibliography.

\chapter{Inserting the index}
After using the \verb"\makeindex" macro and loading the
\texttt{makeidx} package that provides additional indexing
commands, index entries can be created by issuing the \verb"\index"
command. \index{dummy text|(}It is possible to create ranged index
entries, which will encompass a span of text.\index{dummy text|)}
To insert complex typographic material -- such as $\alpha$
\index{alpha@$\alpha$} or \TeX{} \index{TeX@\TeX} --
into the index, you need to specify a text string, which will
determine how the entry will be sorted. It is also possible to
create hierarchal entries. \index{vehicles!trucks}
\index{vehicles!speed cars}

After typesetting the document, it is necessary to generate the
index by running
\begin{center}%
  \texttt{texindy -I latex -C utf8 -L }$\langle$\textit{locale}%
  $\rangle$\texttt{ \jobname.idx}
\end{center}
from the command line, where $\langle$\textit{locale}$\rangle$
corresponds to the main locale of your thesis -- such as
\texttt{english}, and then typesetting the document again.

The \texttt{texindy} command needs to be executed from within the
directory, where the \LaTeX\ source file is located. In Windows,
the command line can be opened in a directory by holding down the
\textsf{Shift} key and by clicking the right mouse button while
hovering the cursor over a directory. Select the \textsf{Open Command
Window Here} option in the context menu that opens shortly
afterwards.

With online services -- such as Overleaf -- the commands are
executed automatically, although the locale may be erroneously
detected, or the \texttt{makeindex} tool (which is only able to
sort entries that contain digits and letters of the English
alphabet) may be used instead of \texttt{texindy}. In either case,
the index will be ill-sorted.

  \makeatletter\thesis@blocks@clear\makeatother
  \phantomsection %% Print the index and insert it into the
  \addcontentsline{toc}{chapter}{\indexname} %% table of contents.
  \printindex

\appendix %% Start the appendices.

\chapter{An appendix}
Here you can insert the appendices of your thesis.

\end{document}
